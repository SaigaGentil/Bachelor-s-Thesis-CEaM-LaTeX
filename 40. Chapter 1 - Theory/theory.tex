% ------------------------------------------------------------------------
% 40. Theory
% ------------------------------------------------------------------------

\chapter{Theory}

In this chapter, key background concepts and methodologies used in the thesis are going to be discussed. The chapter is going to discuss explain what is meant by unmanned aerial system and its components.

The chapter is also going to discuss on the cloud provider, Amazon Web Services (AWS), used to host various components of developed system, simulation and software development tools used, as well as laws and regulations around unmanned aerial systems.

\section{Unmanned Aerial System}
\label{sec:unmanned-aerial-system}

bla bla

\subsection{Unmanned Aerial Vehicle}
bla bla

\subsubsection{Classification of Unmanned Aerial Vehicles}
bla bla


\section{Amazon Web Services}
Amazon Web Services, commonly known as AWS, is a cloud platform provided by Amazon that provides various service offerings such as platform as a service, PaaS, and infrastructure as a service, IaaS\cite{awswhatisaws2022}. AWS makes it easy for developers, engineers and businesses to deploy scalable, resilient, agile and highly available infrastructures for databases, servers, applications, storage, analytics, \textit{et cetera}. AWS offers attractive and cost saving payment strategies of which there are pay-as-you-go, save when you commit, and pay less by using more\cite{awspricing2022}.

Cloud computing is an emerging technilogy that has revolutionized how businesses go online. Cloud computing has been and still is of great use in various industries, including the aerospace and energy industries. Burak et al developed a cloud and edge solution running on AWS that aimed at increasing turbine maintanance inspections' efficiency through automation and a serverless AWS architecture while reducing operations cost\cite{burakawswindfarm2021}. A serverless architecture is a type of architecture where servers' configuration and patching is taken care by the provider, thus allowing developers and engineers to focus on the actual resources, applications, databases \textit{et cetera}, to be deployed. The solution proposed by Burak et al was comprised of drones, machine learning and Internet of Things running on cloud and edge.

The proposed solution in this thesis also takes advantage of what AWS and clound computing offers. Several components, like the ground control system, of the proposed solution are running on AWS. See the high-high-level design in figure \ref{fig:uas-hhld}.

\subsection{Infrastructure as code}
bla bla

\section{Simulation}
bla bla

\subsection{Webots or Ardupilot?}
bla bla

\section{Graphics and software development}
bla bla

\subsection{Microsoft Visual studio code}
\label{subsec:ms-visual-studio-code}
bla bla

\subsection{PyCharm by JetBrains}
\label{subsec:pycharm}
bla bla

\subsection{PhpStorm by JetBrains}
\label{subsec:phpstorm}
bla bla

\subsection{Affinity Designer}
\label{subsec:affinity-designer}
bla bla

\subsection{GitHub}
\label{subsec:github}
bla bla

\subsection{Microsoft Visio}
\label{subsec:ms-visio}
bla bla

\section{Law and regulation}
bla bla


\nomenclature[z-AWS]{AWS}{Amazon Web Services}
\nomenclature[z-IaaS]{IaaS}{Infrastructure as a Service}
\nomenclature[z-PaaS]{PaaS}{Platform as a Service}
\nomenclature[z-ML]{ML}{Machine Learning}
\nomenclature[z-ML]{ML}{Machine Learning}
