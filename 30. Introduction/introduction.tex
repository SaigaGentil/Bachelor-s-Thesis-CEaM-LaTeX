% ------------------------------------------------------------------------
% 30. Introduction
% ------------------------------------------------------------------------

\chapter{Introduction}

Unmanned Aerial Vehicles also known as UAVs or Drones, although hardly a new technology, with the first UAV recorded in history dating back to 1849, have recently gained a lot of attention from various sectors ranging from entertainment to military. This is going to have an impact that cannot be overseen over the coming years as more and more people find uses of UAVs in various applications. UAVs were initially developed to be used for military operations, mainly surveillance, but they were later armed to also enable them to perform long-distance military operations without putting humans at risk. The United States of America has used these types of UAVs mainly in the wars in the Middle East, where UAVs like the General Atomics MQ-9 Reaper also known as Predator B and Northrop Grumman RQ-4 Global Hawk have been widely deployed~\cite{samaanorientxxi2022}.

Despite their use in the military sector, UAVs have also been employed in other sectors such as commercial and entertainment sectors, where UAVs are being used in things like land geography mapping, industrial surveillance, photography and many more. Companies like SZ DJI Technology Co., Ltd. or Shenzhen DJI Sciences and Technologies Ltd. in full, more popularly known as its trade name DJI have had a lot of success in this area, where DJI itself covers (research on the percentage of drones that DJI makes and are on the market). UAVs have also seen great use in the healthcare sector, where companies like Zipline~\cite{droneslevy2022} are implementing an end-to-end supply chain system that employs UAVs to supply and deliver medical supplies to hospitals in rural areas in Rwanda that are hard to reach or inefficient to reach by any other means of delivery. Rwanda has also seen great use of UAVs during the COVID-19 pandemic where UAVs were widely used by the Rwanda’s Ministry of Health and the Rwanda National Police to spread educational knowledge about the COVID-19 to neighborhoods in Kigali (add citation/reference).

As UAVs gain the market, the need to have robust UAV systems also known as UASs becomes eminent. Therefore, in this thesis, focus was put in designing and building a robust, scalable, highly available cloud deployed Unmanned Aerial System, that can easily be integrated with cloud services like Amazon Web Services also known as AWS to provide a solution where UAV pilots can control UAVs from virtually anywhere in the world. The proposed system comprises of a UAV flying with onboard compute that has an LTE datalink to a ground control system also known as GCS, comprised of dashboards and a command-and-control center application running in a highly available and fault tolerant AWS cloud infrastructure. The focus of this thesis is to therefore assess the possibilities of implementing such a solution in an efficient, resilient, reliable, and highly available manner and discuss on the pros and cons of the solution.

The proposed solution was developed following the best industry standards in software development and architecture as is going to be described in detail in this thesis. This thesis is also going to discuss the developments that have already been made in this area as well as areas that need further research and development.


%---------------------------------1.1 Use case---------------------------------%
\section{Use case}

As UAVs emerge, there will be a need to be able to centrally manage a fleet of UAVs. Depending on the UAV use case, operators might need to also control them at a long distance beyond eyesight. A UAV operates as part of a system comprised of multiple other components that support the operation of a UAV. The main components are a Ground Control System, (Research on the main components of a UAV). UAVs can either be Fully autonomous, fully manual, or semi-autonomous. UAVs can also be employed in various use cases, below are various scenarios in which UAVs can be used
\begin{itemize}
    \item Terrain mapping.
    \item Shipping and delivery.
    \item Search and rescue.
    \item Law enforcement.
    \item Military reconnaissance / Surveillance.
\end{itemize}

For a UAV to perform any of the above, it needs to meet certain criteria, a UAV should:
\begin{itemize}
    \item Have onboard computer to process mission commands on the fly.
    \item Have onboard key components like,
          \begin{itemize}
              \item Sensors, depending on the mission.
              \item Cameras.
              \item Battery.
              \item LTE modules or Satnav modules to allow communication with ground control.
          \end{itemize}
    \item Have LTE or Satellite communication to enable the UAV to set up a datalink with the ground control. The UAV would have to send data such as
          \begin{itemize}
              \item	Ground speed.
              \item	Altitude.
              \item	Battery levels.
              \item	Yaw.
              \item	Location.
              \item	Direction.
              \item	Sensor data.
              \item	Send the data frequently for real-time or near real-time communication.
              \item	Be able to react and if necessary, take evasive maneuvers when:
              \item	On collision course.
              \item	The batteries are low on power.
              \item	Out of connectivity range.
          \end{itemize}
\end{itemize}


%---------------------------------1.2 Problem definition---------------------------------%
\section{Problem definition}

\nomenclature[z-UAV]{UAV}{Unmanned Aerial Vehicle}
\nomenclature[z-UAS]{UAS}{Unmanned Aerial System}
\nomenclature[z-GCS]{GCS}{Ground Control Station}
\nomenclature[z-LTE]{LTE}{Long Term Evoluton (Telecommunication)}
\nomenclature[z-DJI]{DJI}{Da-Jiang Innovations}
\nomenclature[z-AWS]{AWS}{Amazon Web Services}