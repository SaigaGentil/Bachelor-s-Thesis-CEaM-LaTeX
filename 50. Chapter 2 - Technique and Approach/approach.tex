% ------------------------------------------------------------------------
% 50. Technique and Approach
% ------------------------------------------------------------------------

\chapter{Methodology and setup}
\label{chap:methodology-and-setup}
In this chapter, the solution is going to be explained in details. The reason behind various design choices is going to be explained elaborately as well as the technical aspects of the solution.

\section{Approach}
In the early stages of the development, the main idea was that the solution had to be:

\begin{itemize}
    \item well planned. Every step taken had to be thought of, usually drawn on a paper to assess its feasibiliy.<ATTACH AN IMAGE OF INITIAL ARCHITECTURE DESIGNS>
    \item well built. The project had to be well structured such that it is intuitive to navigate around source codes. A naming convention was followed throughout the source codes.
    \item easily maintainable. The project was broken down into segments of folders, and the code was built with inheritance.
    \item built with professional, software engineering industry standard toolset, as described in section \ref{sec:tools-used}.
    \item agile. The AWS infrastructure mainly had to be built such that it is easy to deploy and destroy without investing a lot of effort.
\end{itemize}

With the above points in mind, the solution design process started on paper. Figure <REFERENCE AN IMAGE > shows one of the initial designs of the proposed system. After several design iterations, the final design was produced as shown in the high level design on figure <REFENCE THE HLD>. The next step was to bring the design to reality. And that was no easy task.

As the design started being built to reality, it was evident that deploying the whole AWS infrastructure manually from the AWS console, see figure <REFERENCE THE AWS CONSOLE>, was going to be a time consuming, inefficient activity. And this also contracdited the initial idea of building an agile infrastructure that can easily be redeployed in case changes are needed. That is where the thought to build the infrastructure as code using some sort of infrastructure as code (IaC) tool came in. Since the cloud provider of choice was AWS, the best tool for the task was with no doubt the AWS cloud development kit <CITE AWS CDK>.

After the AWS infrastructure was built, the next step was the UAV. The outstanding questions here were:

\begin{itemize}
    \item How is the UAV going to be built?
    \item What is going to be the size?
    \item What is going to be the UAV type? Quadcopter? Fixed wing?
    \item How is the UAV going to be programmed? What programming language should be used?
    \item How is the UAV going to communicate with the rest of the system on AWS?
\end{itemize}

The above questions drove the next development stage. The drone type and size of choice were a small quadcopter. The initial idea was to build an actual quadcopter, therefore the below parts were purchased to build the quadcopter:

\begin{itemize}
    \item Readytosky S500 quadcopter frame with built-in PCB.
    \item Pixhawk 2.4.8 flight controller with 4GB of onboard SD card storage.
    \item Raspberry Pi 3 with 4GB of onboard SD card storage.
    \item Readytosky M8N GPS module built-in compass with GPS antenna mount.
    \item 4 pieces of A2212 1000KV brushless motors.
    \item 4 pieces of 2-6S 30 amps Electronic speed controllers.
    \item 2 pairs of 1238 carbon fiber propellers
\end{itemize}

<INSERT AN IMAGE OF THE ABOVE PARTS>

As the development of the quadcopter went on, it became obvious that this approach was not the best way to go, especially for a proof-of-concept (POC) solution mainly due to how costly it was becoming. At somepoint the wire connectors were even insufficient, and it would take too long to order new ones online. The idea to build the actual physical quadcopter was then put on-hold, and it was decided to rather use simulation tools to simulate the actual quadcopter. Section \ref{sec:software-in-the-loop} elaborates more on how the simulation was set up.

<REVEW THIS SECTION>

\section{Solution description}


\section{Software in the loop simulator}
\label{sec:software-in-the-loop}


\section{Communication}
<Talk about Mavlink... and how mavlink is used in the project>

\section{AWS Network access and security}
One of the challenges with implementing a networked system, especially on cloud platforms like AWS, is ensuring that traffic flows in the expected way with proper security in place. The proposed solution, being a networked solution involving communications to and from various applications, has a rigorous network design. Figure <REFERENCE TO THE NETWORK DESIGN IMAGE> shows how network within the proposed AWS infrastructure was designed.

AWS has a concept of Virtual Private Cloud also known as VPC, which is simply an isolated private network that can be broken down into various subnets depending on the architecture. The proposed solution has one VPC broken down into three subnets; public, private and isolated-private subnets for each availability zone.

\subsection{Public subnet}
\label{public-subnet}
The public subnet in this proposed solution does not contain any resources, except a Network Address Translation or NAT gateway that is used by resources in the private subnet to access the internet. Table \ref{table:public-subnet-inbound} and \ref{table:public-subnet-outbound} show the inbound and outbound traffic rules respectively configured on the public subnet network access control list or NACL.

\begin{table}[H]
    \centering
    \begin{tabular}{|c|c|c|c|c|c|}
        \hline
        \multicolumn{6}{|c|}{Inbound traffic}                               \\
        \hline
        Rule & Type        & Protocol & Port range & Source    & Allow/Deny \\
        \hline
        100  & HTTP (80)   & TCP (6)  & 80         & 0.0.0.0/0 & Allow      \\
        \hline
        110  & HTTPS (443) & TCP (6)  & 443        & 0.0.0.0/0 & Allow      \\
        \hline
        120  & Custom TCP  & TCP (6)  & 1024-65535 & 0.0.0.0/0 & Allow      \\
        \hline
        *    & All IPV4    & All      & All        & 0.0.0.0/0 & Deny       \\
        \hline
    \end{tabular}
    \caption{Public subnet NACL inbound traffic rules}
    \label{table:public-subnet-inbound}
\end{table}

\begin{itemize}
    \item \textbf{Rule 100:} Allows inbound HTTP traffic on port 80 towards any IPv4 address on the internet.
    \item \textbf{Rule 110:} Allows inbound HTTPS traffic on port 443 towards any IPv4 address on the internet.
    \item \textbf{Rule 120:} Allows returning TCP traffic from the internet responding to requests from the subnet. The specified port ranges are ephemeral ports as defined by the Internet Assigned Number Authority or IANA and Internet Engineering Task Force or IETF in their Request for Comments or RFC 6056 document \cite{rfc6056}.
    \item \textbf{Rule *:} Block every other non previously evaluated IPv4 traffic.
\end{itemize}

\begin{table}[H]
    \centering
    \begin{tabular}{|c|c|c|c|c|c|}
        \hline
        \multicolumn{6}{|c|}{Outbound traffic}                                \\
        \hline
        Rule & Type        & Protocol & Port range & Destination & Allow/Deny \\
        \hline
        100  & HTTP (80)   & TCP (6)  & 80         & 0.0.0.0/0   & Allow      \\
        \hline
        110  & HTTPS (443) & TCP (6)  & 443        & 0.0.0.0/0   & Allow      \\
        \hline
        120  & Custom TCP  & TCP (6)  & 1024-65535 & 0.0.0.0/0   & Allow      \\
        \hline
        *    & All IPV4    & All      & All        & 0.0.0.0/0   & Deny       \\
        \hline
    \end{tabular}
    \caption{Public subnet NACL outbound traffic rules}
    \label{table:public-subnet-outbound}
\end{table}

The rules explanation are similar to those for inbound traffic in table \ref{table:public-subnet-inbound}, except that instead of inbound it is outbound.

\subsection{Private subnet}
\label{private-subnet}
Most of the infrastructure components are deployed in the private subnet where only specific traffic from the public and isolated-private subnets are allowed in. In this subnet is where the UAV command and control center user interface is deployed, in containers using the AWS Fargate serverless service. The rules for this subnet have to be carefully defined so that;

\begin{itemize}
    \item Fargate services can pull docker images from docker hub public repositories on the internet.
    \item The UAV, and several command and control application services can talk to each other.
\end{itemize}

Table \ref{table:private-subnet-inbound} and \ref{table:private-subnet-outbound} show the inbound and outbound traffic rules respectively configured on the private subnet network access control list or NACL.

\begin{table}[H]
    \centering
    \begin{tabular}{|c|c|c|c|c|c|}
        \hline
        \multicolumn{6}{|c|}{Inbound traffic}                                 \\
        \hline
        Rule & Type        & Protocol & Port range & Source      & Allow/Deny \\
        \hline
        100  & HTTP (80)   & TCP (6)  & 80         & 0.0.0.0/0   & Allow      \\
        \hline
        110  & HTTPS (443) & TCP (6)  & 443        & 0.0.0.0/0   & Allow      \\
        \hline
        120  & Custom TCP  & TCP (6)  & 1024-65535 & 0.0.0.0/0   & Allow      \\
        \hline
        130  & Custom TCP  & TCP (6)  & 3306       & 10.0.4.0/28 & Allow      \\
        \hline
        140  & Custom TCP  & TCP (6)  & 3306       & 10.0.5.0/28 & Allow      \\
        \hline
        *    & All IPV4    & All      & All        & 0.0.0.0/0   & Deny       \\
        \hline
    \end{tabular}
    \caption{Private subnet NACL inbound traffic rules}
    \label{table:private-subnet-inbound}
\end{table}

\begin{itemize}
    \item \textbf{Rule 100:} Allows inbound HTTP traffic on port 80. This is so that the AWS Elastic Container Service or ECS tasks can pull images from the public Dockerhub registry.
    \item \textbf{Rule 110:} Allows inbound HTTPS traffic on port 443.
    \item \textbf{Rule 120:} Allows returning TCP traffic from the internet responding to requests from the subnet.
    \item \textbf{Rule 130 and Rule 140:} Allows inbound traffic on port 3306 from MySQL database running in the AWS Relational Database Service or AWS within the isolated-private subnets of both the Availability Zones.
    \item \textbf{Rule *:} Blocks every other non previously evaluated IPv4 traffic.
\end{itemize}

\begin{table}[H]
    \centering
    \begin{tabular}{|c|c|c|c|c|c|}
        \hline
        \multicolumn{6}{|c|}{Outbound traffic}                                \\
        \hline
        Rule & Type        & Protocol & Port range & Destination & Allow/Deny \\
        \hline
        100  & HTTP (80)   & TCP (6)  & 80         & 0.0.0.0/0   & Allow      \\
        \hline
        110  & HTTPS (443) & TCP (6)  & 443        & 0.0.0.0/0   & Allow      \\
        \hline
        120  & Custom TCP  & TCP (6)  & 1024-65535 & 0.0.0.0/0   & Allow      \\
        \hline
        130  & Custom TCP  & TCP (6)  & 3306       & 10.0.4.0/28 & Allow      \\
        \hline
        140  & Custom TCP  & TCP (6)  & 3306       & 10.0.5.0/28 & Allow      \\
        \hline
        *    & All IPV4    & All      & All        & 0.0.0.0/0   & Deny       \\
        \hline
    \end{tabular}
    \caption{Private subnet NACL outbound traffic rules}
    \label{table:private-subnet-outbound}
\end{table}

\begin{itemize}
    \item \textbf{Rule 100:} Allows outbound HTTP traffic on port 80 towards any IPv4 address.
    \item \textbf{Rule 110:} Allows outbound HTTPS traffic on port 443 towards any IPv4 address.
    \item \textbf{Rule 120:} Allows all outbound response TCP traffic.
    \item \textbf{Rule *:} Blocks every other non previously evaluated IPv4 traffic.
\end{itemize}

<TALK ABOUT SECURITY GROUPS>

\subsection{Isolated-private subnet}
\label{isolated-private-subnet}

The isolated-private subnet hosts the MySQL database running in AWS Relational Database Service. This subnet only talks to the private subnet, and has no direct connection to the internet. This improves the infrastructure security through not exposing the database directly to the internet.

<ADD NETWORK FLOW DESIGNS>

Describe the solution on a higher level. Discuss HLDs.

\nomenclature[z-AZ]{AZ}{Availability Zone}
\nomenclature[z-NAT]{NAT}{Network Address Translation}
\nomenclature[z-VPC]{VPC}{Virtual Private Cloud}
\nomenclature[z-HTTP]{HTTP}{Hypertext Transfer Protocol}
\nomenclature[z-HTTPS]{HTTPS}{Hypertext Transfer Protocol Secured}
\nomenclature[z-ECS]{ECS}{Elastic Container Service}
\nomenclature[z-NACL]{NACL}{Network Access Control List}
\nomenclature[z-EC2]{EC2}{Elastic Cloud Compute}
\nomenclature[z-TCP]{TCP}{Transmisison Control Protocol}
\nomenclature[z-RDS]{RDS}{Relational Database Service}
\nomenclature[z-ALB]{ALB}{Application Load Balancer}
\nomenclature[z-RDS]{RDS}{Relational Database Service}
\nomenclature[z-IANA]{IANA}{Internet Assigned Number Authority}
\nomenclature[z-IETF]{IETF}{Internet Engineering Task Force}
\nomenclature[z-RFC]{RFC}{Request for Comments}
\nomenclature[z-PCB]{PCB}{Printed Circuit Board}
\nomenclature[z-SD]{SD (as in SD card)}{Secure Digital}
\nomenclature[z-GPS]{GPS}{Globa Positioning System}
