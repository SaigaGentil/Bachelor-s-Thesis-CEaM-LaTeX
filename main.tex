%%%---------------------------------------------------------------%%%
%%% Wyzsza Szkola Gospodarki Bachelor's Thesis                    %%%
%%% Prepared by Bruno Axel Kamere                                 %%%
%%% Inspiration Artur M. Brodzki & Piotr Woźniak's WUT template.  %%%
%%% Computer Engineering and Mechatronics Department              %%%
%%% Wyzsza Szkola Gospodarki w Bydgoszczy, 2022                   %%%
%%%---------------------------------------------------------------%%%

\documentclass[
    left=2.5cm,         % Sadly, generic margin parameter
    right=2.5cm,        % doesnt't work, as it is
    top=2.5cm,          % superseded by more specific
    bottom=3cm,         % left...bottom parameters.
    bindingoffset=6mm,  % Optional binding offset.
    nohyphenation=false % You may turn off hyphenation, if don't like.
]{config/config-thesis}

\langeng % Use English language
\graphicspath{img/} % Image path
\addbibresource{bibliography.bib}

\begin{document}

\EngineerThesis
\institute{Informatics and Mechatronics}
\field{Computer Engineering and Mechatronics}
\title{Cloud Managed Unmanned Aerial System}
\author{Bruno Axel Kamere}
\studentnumber{030756}
\seminarlecturer {Szychta Elżbieta, prof. dr hab. inż.}
\supervisor{Ocetkiewicz Tomasz, mgr inż.}
\date{\the\year}
\maketitle

%--------------------------------
% Dedication
%--------------------------------

%--------------------------------
% Acknowledgements
%--------------------------------

%--------------------------------
% Abstract
%--------------------------------
\newpage
\abstract \lipsum[1-3]
\keywords AWS, UAV, UAS

%--------------------------------
% Declaration of author's will
%--------------------------------
\cleardoublepage
\pagestyle{plain}
\makeauthorship

%--------------------------------
% Table of Contents
%--------------------------------
\cleardoublepage
\tableofcontents

%--------------------------------
% Chapters - Introduction
%--------------------------------
\cleardoublepage
\pagestyle{headings}

\input{chapters/1-preface}

%--------------------------------
% Chapters - Theory
%--------------------------------

%--------------------------------
% Chapters - Technique and Resources
%--------------------------------

%--------------------------------
% Chapters - Setup
%--------------------------------

%--------------------------------
% Chapters - Discussion
%--------------------------------

%--------------------------------
% Chapters - Conclusion
%--------------------------------

%--------------------------------
% Chapters - List of Figures
%--------------------------------

%--------------------------------
% Chapters - Nomenclature
%--------------------------------

%--------------------------------
% Chapters - References
%--------------------------------

%--------------------------------
% Chapters - Appendix
%--------------------------------
\newpage
\section{Summatio}

\end{document}
